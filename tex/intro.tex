The Standard Model (SM) of particle physics has been very successful at describing the interactions of fundamental particles. With the discovery of the Higgs boson, the SM is complete and may be the correct theory up to the energy scale of gravity. The Large Hadron Collider (LHC) was built in order to probe this. The SM and look for solutions to some of the unknown issues in particle physics that may involve physics beyond the SM.

Despite the overwhelming success of the SM, some uncertainty remains in predictions in the Quantum Chromodynamics (QCD) sector. Due to its non-pertubative nature, QCD has not been as precisely measured as other parts of the SM. In particular, since the top quark was only discovered in the late 1990s, its decays and properties have not been as studied as thoroughly. 

The top quark plays a special role in the SM and in searches for
physics beyond the SM .  Its large mass means its coupling to the
Higgs Boson is large.  This high mass, together with the presence of 
charged leptons, missing energy and $b$-jets as top decay products,
make the top a primary source of background in many searches for new physics.
For these reasons, accurate modeling of the properties of top quark
events is an important part of the LHC program.

The inclusive top quark pair cross-section was measured using the same events as this thesis in Ref.~\cite{xsec}. This thesis builds upon this work by studying the production of jets in these events, which provides additional information about the production of QCD radiation in association with top quarks. The data are fully corrected to allow for direct comparisons to predictions by theoretical models. This will allow for improved agreement of future QCD predictions with LHC data.

This thesis is structured as follows. Chapter~\ref{ch:theory} reviews the SM and theoretical predictions for top quarks. Chapter~\ref{ch:pheno} outlines simulation techniques and jet algorithms necessary used in the modeling of hadron collisions. Chapter~\ref{ch:strategy} describes the motivation and overall strategy for the thesis. Chapter~\ref{ch:atlas} gives an overview of the ATLAS experiment. Chapter~\ref{ch:objects} explains the procedures used to reconstruct physics objects. Chapter~\ref{ch:event} outlines the data samples used and gives the requirements for selecting events. Chapter~\ref{ch:extrajets} describes the definition for extra jets and compares the simulation predictions to the data. Chapter~\ref{ch:unfolding} gives the unfolding procedure used to correct the data for experimental biases. Chapter~\ref{ch:syst} outlines the sources and evaluation of systematic uncertainty. Chapter~\ref{ch:results} presents the fully corrected results with comparisons between simulation and data.
