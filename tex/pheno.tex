\section{Simulation of hadron collisions}

In order to analyze complicated processes at huge machines like the LHC, software that virtually reproduces the physics experiment is necessary. In order to account for stochastic effects, Monte Carlo (MC) methods are used to simulate a large number random events.  In the actual experiment, ATLAS detects collisions produced by the LHC and stores these \emph{events} in a data acquisition system. In the virtual simulation, event generators such as \hw~\cite{Herwig} and \py\cite{Pythia} produce final state particles. These particles are then run through a detector simulation of ATLAS built with \textsc{Geant 4}. The simulated and actual detector signals can then share the same event reconstruction framework and analysis. This allows a clear understanding of how the input physics is distorted step-by-step as it goes through the detector and reconstruction. 

The distortions resulting from detector imperfections and reconstruction are particularly important to this thesis, so the following chapters will employ specific terms to refer to different aspects of the simulation process. \emph{Generator-level} or \emph{truth} particles refer to those produced by the MC generator before any detector interactions. \emph{Detector-level} or \emph{reconstructed} particles refer to those that have gone through the detector simulation and been reconstructed. 

To go backwards from the (distorted) reconstructed to truth particles, the distortions introduced by the detector must be reversed. This correction process is referred to as \emph{unfolding} and is further discussed in Chapter~\ref{ch:unfolding}. Unfolding is necessary to compare actual data measured with the detector to theoretical predictions produced by generators.
\section{Monte Carlo generators}
In MC event generators, events are produced step-by-step, with random numbers pulled from quantum mechanical probability distributions at various stages. Averaging over a large number of events gives the expected final distribution of events. The goal is to start with a QFT matrix element in a form similar to Equation~\ref{eq:ab} and produce final state, stable particles that can be measured with a detector. Some of the available event generators go through the complete simulation process while others can only handle some of the steps.

The steps to generate an event can be summarized as follows:
\begin{itemize}
\item
\end{itemize}


\section{Jets}
why antikt is good